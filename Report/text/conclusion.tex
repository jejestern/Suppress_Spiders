\section{Conclusion}
\label{sec:conclusion}
To suppress the spiders in astronomical data we used Fast Fourier transformation. In order to do so it is a good method to warp the image data to the $r$-$\varphi$ plane. In this plane the radial intensity structures are then all parallel to the new $y$-axis and now the radial intensity drop-off due to the star can easily be filtered out.\\
In our work we used data from HD142527 and we looked at the radial regime from $R=254-454$, where the two ghosts are situated. We calculated their aperture flux to check that our suppression did not accidentally filter out round objects. Additionally, we also inserted a simulated PSF to investigate the effect on fainter objects. \\
We decided to suppress the spiders by dividing the low frequencies by a Gaussian profile. Using a division allows us the use of a Gaussian profile as an approximation for the real signal, which is a lot more complicated. Additionally, we can use the same intensity and width of the Gaussian profile for all images. \\
However, we need to limit the angular range, so that we do not divide by small numbers. Another problem of the division is that it is actually not the right approach, if we consider the theory. Dividing means that we treat the spiders as if they were a convolution on the image, but in reality they are added on top of the image data. This means that we should subtract the frequencies caused by the spiders. The problem about the subtraction is that we cannot do useful approximations. Also the parameters are a lot more sensible and need to be chosen precisely. \\
Although the suppression by dividing the low frequencies by a Gaussian profile is theoretically not the right approach, we still receive a result, where the spiders are suppressed, at least a little and we do not loose much information about the objects at which we are interested. However the suppression only works well around the center of the chosen radius range. Also the resulting suppression is quite small, so that it is questionable if it makes sense to do it. If one decides to do suppression by division, then one should only suppress the central radial frequency, since the suppression by other frequencies does not lead to success.\\ 
Suppressing the spiders by subtracting the right frequencies does not lead to success. The problem is that the frequencies caused by the spiders and the ones caused by the interesting objects in the image overlay. It is therefore necessary to know exactly how the Fourier transform of the spiders looks like. However this Fourier transform is highly sensible on the position and the separation between the spiders, as well as on the different widths and intensities. We can not find these parameters without risking to loose important information in our data. Also when we know all these parameters we can also directly subtract the spiders from the flattened image without transforming it into the frequency plane.\\ 
In section \ref{sec:pointlike} we saw that the Fourier transform of a PSF has this characteristic elliptic ring-shaped structure. Maybe it would be possible to enhance the signal of point-like sources by enhancing the frequencies inside these rings. \\
In conclusion we can say that suppressing the spiders in astronomical data by suppressing the respective frequencies in the frequency plane is not a success. 
 