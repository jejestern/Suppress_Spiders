\section{Conclusion}
To suppress the spiders in astronomical data we used Fast Fourier transformation. In order to do so it is a good method to warp the image data to the $r$-$\varphi$ plane. In this plane the radial intensity structures are then all parallel to the new $y$-axis as the spider and now that the radial intensity drop-off due to the star can be filtered out easily.\\
In our work we used data from HD142527 and we look at the radial regime from $R=254-454$. In this regime the two ghosts are situated, we calculated there aperture flux to check that our suppression did not accidentally filter out round objects. Additionally we also inserted a simulated PSF to investigate the effect on fainter objects. \\
We decide to suppress the spiders by dividing the low frequencies by a Gaussian profile. The advantage of the division is that we can use the Gaussian profile as an approximation of the real signal, which is a lot more complicated. Additionally we can use the same intensity and width of the Gaussian profile for all images, since the suppression by division is not sensible on these parameters. However we need to limit the angular range, so that we do not divide by small numbers. An other problem of the division is that it is actually not the right approach, if we consider the theory. Dividing means that we treat the spiders as if they were a convolution on the image, but in reality they are added on top of the image data. This means that we should subtract the frequencies caused by the spiders. The problem about the subtraction is that we cannot do all these approximation, like approximating it by a Gaussian. Also the parameters are a lot more sensible and need to be chosen precisely. \\
Although the suppression by dividing the low frequencies by a Gaussian profile is theoretically not the right approach, we still receive a result, where the spiders are suppressed, at least a little and we do not loose much information of the objects in which we are interested. However the suppression only works well around the center of the radius range chosen. Also the resulting suppression is quite small, so that it is questionable if it makes sense to do it. If one decides to do suppression by division, then one should only suppress the central radial frequency, since the suppression by other frequencies does not lead to success.\\ 
It would make sense to try to suppress the spiders by subtracting certain frequencies in a future work, also if this means that one needs to determine the parameters really precisely and have individual parameters for each image data. 
 