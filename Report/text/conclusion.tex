The program uses Fourier transformation to suppress the spiders in astronomical data.
Here we use data from HD142527 and used the aperture fluxes of the ghosts and 
a simulated PSF to adjust the parameters for the suppression until they fit our
expectations. 
We use a Gaussian profile for the suppression, but instead of subtracting (how
it should be) we divide by the profile. This is a really stable action, but it 
is not correct, since dividing is like treating the spiders as if they are 
convolutions in the image plane, which they aren't. Still we receive some kind 
of a good result where the spiders are suppressed. The cool thing about the 
result however is, that the image gets a smooth background.


In conclusion, we find that 
Differences: Images have different intensities -> angular division width needs to be adapted to the overall intensity -> Is it not really worth it, since the results are not really great

if this is used, than only suppress central radial freq, and leave lower frequencies in peace...