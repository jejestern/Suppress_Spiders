\section{Fourier Transformation}
Our goal is to decrease the intensity of the spyders and speckles in the data. In our data from HD142527 the main disturbing effects are the spyders. With the transformation to the $r$-$\varphi$ plane the spyders now are parallel to the $r$-axis. When comparing different data sets from the same observation we find that the spyders change there position, but the distance between them stays the same. It is a periodic pattern. This brings up the idea, if the spyders are represented by a set of frequencies in the frequency plane. If this is the case, the suppressing of some of the frequencies in the frequency plane would result in a suppressing of the spyders in the image plane. The advantage of this method would be that it could be applied to the complete data set and one can ignore the fact that the spyders wander along the $\varphi$-axis. Also we should be able to suppress the spyders without destroying the information below them.\\
In the following we are going to investigate the properties of a fourier transformation and the effect of suppressing certain frequencies on the image plane.\\

\subsection{Fourier Transformation of some Basic Features}
