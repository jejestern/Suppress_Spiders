\section{Introduction}
The question whether there is life on other planets then earth has always been of interest form humans. To find an answer to this question scientist try to find possible habitable planets. Since it is easier to search for life which is already known an approach is to search for earth-like planets.\\
A problem in the search for habitable planets via direct imaging is mainly that it is assumed that they need to be already older, since young planets are hot and old planets are harder to detect. The temperature of a young giant planet is between $T \approx 1000$ - $2000$ K this means that it will emit mainly in the near-infrared and the contrast which is required to resolve young planets $C = F_{\mathrm{pl}}/F_{\mathrm{star}} \approx 10^{-5 \pm 1}$ \cite{Hunziker2020}. Whereas an old planet is a lot colder and the light it emits is in the visual to near-infrared, since it is caused by the scattering of the host-stars light. Therefore the contrast required to resolve an old planet is a lot smaller, namely around $C \lessapprox 10^{-7}$ \cite{Hunziker2020}.\\
With the current instruments available it is still very challenging to reach this small contrast. Therefore most planets which have been found so far, which are potentially habitable are close to our solar system and/or their host star is less luminous than our sun.\\
In order to be able to detect planets with small contrasts we can either improve our instruments and measurement methods or we can try to improve the quality of the already taken data already with data analysis. In this report we are going to focus on the later. \\
In the following we are going to describe the possibility to suppress the spiders in astronomical data with the help of Fast Fourier transformation. In section \ref{sec:aperture_phot} we describe how the aperture flux is calculated and we look at the aperture flux of the ghosts in the image data of HD142527. HD142527 is a binary star which is still quite young. Therefore the stars are surrounded by a protoplanetary disk and the system can be used to investigate the planet formation processes inside the protoplanetary disk \cite{HD142527}.\\
The transformation of the image data into the $r$-$\varphi$ plane is described in section \ref{sec:interpolation}, where we also investigate the effect this transformation has on the data. Then we subtract the radial intensity drop off from the transformed image as it is explained in section \ref{sec:radial_intensity}. In section \ref{sec:fourier} we take a closer look at the Fourier transformation and how the frequency plane of specific features looks like. Here we focus primarily on spider like features. After we have gained an understanding of the frequency plane we use this knowledge in section \ref{sec:suppression_divide} to suppress certain frequencies in order to get back an image where the spiders have vanished. At the end we summarize our results in section \ref{sec:suppression_divide}.\\
The scripts used for the simulations and calculations are written in python and can be found 
at HERE THE LINK TO THE REPOSITORY STILL NEEDS TO BE ADDED ????????????????
