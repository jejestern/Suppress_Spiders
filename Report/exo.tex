\documentclass[a4paper,10pt,oneside, fleqn]{article}
% Druckbereich: \areaset[BCOR]{textwidth}{textheight}
% BCOR ist "Binding Correction", also wieviel Innenrand verloren geht
% A4 hat 297mm x 210mm
% wenn keine Marginalien, dann ist Breite 15cm vielleicht besser
%\areaset[0cm]{15cm}{25cm}


\usepackage[english,mt]{ethidsc}    % deutsche Übersetzungen und Wortumbrüche
\usepackage{mcode}
\usepackage{subfigure}

   
% Page header (don't change)____________________________________________________
\setlength{\parindent}{0em}                 % Disable parindent
\rhead[\nouppercase{\rightmark}]{\thepage}  % Special headings
\lhead[\thepage]{\nouppercase{\leftmark}}   % Special headings
\cfoot{} 
 
%%% hier können noch viel viel mehr Einstellungen kommen

%%% Title page
\title{Using Fast Fourier Transformation to Suppress Spiders in Astronomical Data}

\studentA{Jennifer Studer}
\ethidA{16-915-928}
\semesterA{2}
\emailA{studerje@student.ethz.ch}

\supervision{Hans Martin Schmid\\ Christian Tschudi}
\date{\today}

 
%%%% hier beginnt der Inhalt %%%%%%%%%%%%%%%%%%%%%%%%%%%%%%%%%%%%%%%%%%%%%%%%
\begin{document}

\maketitle

\pagestyle{fancy}               	% Fancy headings
\pagenumbering{arabic}

\vspace*{\fill}
\begin{abstract}
To suppress the spiders in astronomical data we use Fourier theory. We first transform the image data to the $r$-$\varphi$ plane so that the spiders are parallel to each other. Then we simulate the spiders to investigate the frequency space of spiders only. With the knowledge gained by the simulations we suppress some specific patterns in the frequency plane of the image data which are caused by the spiders and then transform the result back into the image plane. We receive an image where the spiders are slightly suppressed. 
\end{abstract}
\vspace*{\fill}
\newpage

\tableofcontents
\newpage

\section{Introduction}
The question whether there is life on other planets then earth has always been of interest for humans. To find an answer to this question scientist try to find possible habitable planets. Since it is easier to search for life which is already known an approach is to search for earth-like planets.\\
A problem in the search for habitable planets via direct imaging is mainly that it is assumed that they need to be already older, since young planets are hot and old planets are harder to detect. The temperature of a young giant planet is between $T \approx 1000$ - $2000$ K this means that it will emit mainly in the near-infrared and the contrast which is required to resolve young planets $C = F_{\mathrm{pl}}/F_{\mathrm{star}} \approx 10^{-5 \pm 1}$ \cite{Hunziker2020}. Whereas an old planet is a lot colder and the light it emits is in the visual to near-infrared, since it is caused by the scattering of the host-stars light. Therefore the contrast required to resolve an old planet is a lot smaller, namely around $C \lessapprox 10^{-7}$ \cite{Hunziker2020}.\\
With the current instruments available it is still very challenging to reach this small contrast. Therefore most planets which have been found so far, which are potentially habitable are close to our solar system and/or their host star is less luminous than our sun.\\
In order to be able to detect planets with small contrasts we can either improve our instruments and measurement methods or we can try to improve the quality of the already taken data with data analysis. In this report we are going to focus on the later. \\
In the following we are going to describe the possibility to suppress the spiders in astronomical data with the help of Fast Fourier transformation. In section \ref{sec:aperture_phot} we describe how the aperture flux is calculated and we look at the aperture flux of the ghosts in the image data of HD142527. HD142527 is a binary star which is still quite young. Therefore the stars are surrounded by a protoplanetary disk and the system can be used to investigate the planet formation processes inside the protoplanetary disk \cite{HD142527}.\\
The transformation of the image data into the $r$-$\varphi$ plane is described in section \ref{sec:interpolation}, where we also investigate the effect this transformation has on the data. Then we subtract the radial intensity drop off from the transformed image as it is explained in section \ref{sec:radial_intensity}. In section \ref{sec:fourier} we take a closer look at the Fourier transformation and how the frequency plane of specific features looks like. Here we focus primarily on spider like features. After we have gained an understanding of the frequency plane we use this knowledge in section \ref{sec:suppression_divide} to suppress certain frequencies in order to get back an image where the spiders have vanished. At the end we summarize our results in section \ref{sec:suppression_divide}.\\
The scripts used for the simulations and calculations are written in python and can be found 
at HERE THE LINK TO THE REPOSITORY STILL NEEDS TO BE ADDED ????????????????

\section{Aperture Photometry}
In order to determine the flux of different objects in astrophysics, like stars and planets, aperture photometry can be used. This method sums the counts of the pixels inside a certain aperture around the star. In our case this aperture is usually a circle. In order to account for the background noise an annulus around the aperture is taken and the mean of the summed up pixels inside the annulus is subtracted from the apertures pixels. The flux of this aperture is then given by \cite{Gisin}
\begin{equation}
	F_{ap} = F_{tot} - n_{px} \langle F_{bg} \rangle ,
\end{equation}
where $F_{tot}$ is the total flux inside the aperture (sum up the pixel values inside the aperture), $n_{px}$ is the number of pixels inside the aperture and $\langle F_{bg} \rangle$ is the mean background per pixel. This mean background per pixel is defined through the annulus and calculated from
\begin{equation}
	\langle F_{bg} \rangle = \frac{1}{m} \sum_{i=1}^{m} c_{i} ,
\end{equation}
where $m$ is the number of pixels in the annulus and $c_{i}$ the respective pixel value.
Figure \ref{fig:aperture_ex} shows an example for a possible aperture and an annulus around a star, which can be used to do an aperture photometry.
\begin{figure}[H]
	\centering
		\includegraphics[width=0.9\textwidth]{pics/aperture_example.pdf}
		\caption{An aperture photometry for the star in the center, where the red circle indicates the aperture used and the two blue circles define the annulus used for the background subtraction.}
		\label{fig:aperture_ex}
\end{figure}
From the figure you can see that we chose the annulus not directly after the aperture, but this is just one way to do it. One could also choose the annulus directly after the aperture or choose a different distance between the annulus and the aperture. However the annulus should give a good approximation for the background inside the aperture and therefore it should not be too far away from the aperture. We chose a small distance between the aperture and the annulus of 4 pixels, because we wanted that as little starlight (in this case) as possible is included in the annulus. If we plot the counts per pixels which are included in a certain radius around the star, as it is done in figure \ref{fig:annulus_radi} we see that after around 10 pixels the increase is decreasing rapidly. That is where there is only little starlight left. This is also why we choose the annulus to go from a radius of 10 to 15 pixels. 
\begin{figure}[H]
	\centering
		\includegraphics[width=0.8\textwidth]{pics/CountsPerRadius.pdf}
		\caption{The total flux of the star is calculated for different radii and plotted. This shows that after a radius of 10 pixels the contribution from the star is almost gone.}
		\label{fig:annulus_radi}
\end{figure}
Additionally we choose the radius of the aperture to be 6 pixels.

\subsection{Aperture photometry and ghosts}
In order to detect exoplanets one can use aperture photometry. In this subsection we are going to do these steps for the two ghosts which we have in our data from the circumstellar disk HD142527, in order to demonstrate how this could be done for an exoplanet and to learn more about ghosts.\\
A ghost is a copy of the star, which is created by the back-reflection of the star on optical components of the telescope. Figure \ref{fig:ghosts} is an image of HD142527, where the two ghosts are indicated. We see that the ghost on the top right (we will call this ghost 1) is brighter than the ghost on the bottom left (ghost 2). 
\begin{figure}[H]
	\centering
		\includegraphics[width=1.3\textwidth]{pics/Ghosts.pdf}
		\caption{An image of the circumstellar disk HD142527, where the two ghosts are indicated.}
		\label{fig:ghosts}
\end{figure}
If we want to confirm the signal from our ghost or also from other objects like exoplanets, which usually can not be seen by eye, we use the signal to noise ratio $S/N$. This means we do aperture photometry for several points around the star which are at the same separation from our star as the ghost (or exoplanet etc.), as in figure \ref{fig:ap_phot_gh1}. From this we can calculate the standard deviation $\sigma$ from the mean of all aperture fluxes and the signal to noise ratio. If the signal to noise ration is larger than the standard deviation, this means that we have a source (ghost, exoplanet) at this position. 
\begin{figure}[H]
	\centering
		\includegraphics[width=0.9\textwidth]{pics/aperture_photometry_19_ghost1.pdf}
		\caption{In order to confirm the signal from the ghost, we do aperture photometry for several points around the star which are at the same distance from the star as the ghost. We then calculate the standard deviation of all the aperture fluxes and from there find the signal to noise of the ghost's aperture. If the signal to noise is larger than the standard deviation, the position of the ghost is confirmed.}
		\label{fig:ap_phot_gh1}
\end{figure}
The standard deviation is calculated by
\begin{equation}
	\sigma = \sqrt{\frac{1}{k-2} \sum_{ap=2}^{k} (F_{ap} - F_{mean})^2} ,
\end{equation}
where the aperture of the ghost is at $ap=1$, $k$ is the number of apertures and $F_{mean}$ is the mean flux of the apertures (without the aperture of the ghost) given by
\begin{equation}
	F_{mean} = \frac{\sum_{ap=2}^{k} F_{ap}}{k-1} .
\end{equation}
From this we can find the signal to noise ratio as
\begin{equation}
	S/N = \frac{F_1 - F_{mean}}{\sigma} .
\end{equation}

Als nächstes: results und wie viel schwächer ist der ghost

\section{Transformation to the r-phi plane}
We have now seen that the ghosts are about $10^{-4}$ times less bright then their star, but potential exoplanets are even less bright. A Jupiter like exoplanet would be $10^{-9}$ less bright (reflecting light) and an Earth like even only $2 \cdot 10^{-10}$. Therefore the data must be really sensitive, with a high contrast and a high spatial resolution. As we can see in figure \ref{fig:ghosts} the star produces strong spiders and speckles which have similar brightness as the ghosts or are even brighter. Therefore exoplanets which are situated in this regimes cannot be detected, unless we are able to take out the signals of the spiders and speckles without taking away other signals, like the ones from exoplanets.\\
If we take a closer look at figure \ref{fig:ghosts} we observe that the structure of the spiders and the speckles are radially oriented around the star. In order to get rid of this effects it might be a good idea to transform the image into the r-phi plane, where we define the star to be at radius zero. After the transformation the spiders and speckles are distributed along the phi axis and they become weaker along the r axis.\\
Lets have a look at how one can transform the image into the r-phi plane. This kind of transformation is called image warping in image processing. In our case the image warping is based on a specific transformation, namely the transformation from cartesian to polar coordinates and is therefore also called polar-cartesian distortion.\\
In general an image warping is based on a transformation $T: \mathbb{R}^2 \rightarrow \mathbb{R}^2$, such that 
\begin{equation}
	\vec{x} \mapsto \vec{u} = \begin{pmatrix} T_u(x,y) \\ T_v(x,y) \end{pmatrix},
\end{equation}
where $\vec{x} = \begin{pmatrix} x \\ y \end{pmatrix}$ and $\vec{u} = \begin{pmatrix} u \\ v \end{pmatrix}$. When we want to warp an image $f$ into an image $g$ we do the following calculation
\begin{equation}
	g(\vec{u}) = g(T(\vec{x})) = f(\vec{x}).
\end{equation}
This means that at pixel $\vec{u}$ the computed image $g$ has the same intensity as the original image $f$ at pixel $\vec{x}$. \cite{ImageWarping}\\
Next:\\
- Keep in mind: not bilinear and Before we can do the distortion we have to define our coordinate systems. \\
- how I did it (insert a little code fragment of the warping to r-phi) 
- what about warping back? (no code fragment)
- data examples
- circle -> to explain chosen length of phis
- was the aperture preserved? 

\section{Radial Intensity drop off}
Due to the star in the center of the image, there is an intensity decrease in radial direction. This light coming from the star makes it harder to find other objects and structures close to it. Therefore we want to get rid of it. This is a lot easier in the $r$-$\varphi$ plane, where the intensity decrease is parallel to the $r$ axis as we can see in figure \ref{fig:warping_R150-300} b) and this drop-off can be described by an exponential decrease. In figure \ref{fig:mean_angular_intensity_R150_300} the mean angular intensity values are plotted against the radius and the data is fit by an exponential.
\begin{figure}[H]
	\centering
		\includegraphics[width=0.9\textwidth]{pics/mean_angular_intensity_R150_300.pdf}
\caption{The intensity drop off due to the light from the star at radius zero can be described by an exponential fit.}
\label{fig:mean_angular_intensity_R150_300}
\end{figure}
The exponential fit describes the intensity drop off in a good way and by subtracting it from the $r$-$\varphi$ plane image we get an image where the mean intensity in radial direction is more or less constant. Figure \ref{fig:flatten_R150-300} shows the effect of this subtraction and we call the resulting image the flatten image.  
\begin{figure}[H]
	\centering
		\subfigure[]{\includegraphics[width=1.0\textwidth]{pics/HDwarped_R150_R300_15.pdf}}
		\subfigure[]{\includegraphics[width=1.0\textwidth]{pics/HDflatten_R150_R300_4.pdf}}
\caption{After the image transformation into the $r$-$\varphi$ plane (a), the image is flattened by subtracting the intensity which comes from the star at radius zero (b).}
\label{fig:flatten_R150-300}
\end{figure}
We insert a model planet (a faint copy of the star) into the image before the flattening of the image to make sure, that the aperture flux is not affected by the flattening. We find that the aperture flux change due to the flattening is only $0.03 \%$. So we can say that the aperture flux is not affected by this procedure. 
\section{Fourier Transformation}
Our goal is to decrease the intensity of the spiders and speckles in the data. In our data from HD142527 the main disturbing effects are the spiders. With the transformation to the $r$-$\varphi$ plane the spiders now are parallel to the $r$-axis. When comparing different data sets from the same observation we find that the spiders change there position, but the distance between them stays the same. It is a periodic pattern. This brings up the idea, if the spiders are represented by a set of frequencies in the frequency plane. If this is the case, the suppression of some of the frequencies in the frequency plane would result in a suppression of the spiders in the image plane. The advantage of this method would be that it could be applied to the complete data set and one can ignore the fact that the spiders wander along the $\varphi$-axis. Also we should be able to suppress the spiders without destroying the information below them.\\
In the following we are going to investigate the properties of a Fourier transformation on some specific image structure as lines, beams and point-like sources.\\

\subsection{Lines and Beams}
\label{FFT_Lines_Beams}
We first want to investigate the effect of some simple signals in the image plane on the frequency plane. We choose these signals similar to the shape of the spiders or to the shape of a potential exoplanet, with the goal to identify similar characteristics in the frequency plane of the data.\\
Firstly, we transform a single line (has the width of one pixel) in vertical or horizontal direction to the Fourier plane. A single line in vertical direction ($y$-axis) means that we have a constant signal along the $y$-axis, but we have a non-constant signal in $x$ direction, namely a one pixel wide peak at the $x$ position of the line. So we expect that all $y$-axis frequencies in the Fourier transformed image to be zero, this means that only at $y$ frequency zero we have non-zero values which describe the periodicity in $x$ direction.\\
As we can see from figure \ref{fig:fft_line} the transformation of a single line results into a single line in the frequency plane. The line in the frequency plane is perpendicular to the line in the image plane and is located in the center. This confirms our expectations.\\
Since we plot our results with a logarithmic scale, we need to keep in mind that $\log(0) = -\infty$ . In order to be able to plot our results we added a small value $\epsilon$ to our Fourier transformed results just before plotting.
\begin{figure}[H]
	\centering
		\subfigure[]{\includegraphics[width=1.0\textwidth]{pics/fft_simulationoneline.pdf}}
		\subfigure[]{\includegraphics[width=1.0\textwidth]{pics/fft_simulationoneline_horizontal.pdf}}
\caption{The image of a vertical line (a) and of a horizontal line (b) (images on the left side) are transformed to the frequency plane (images on the right side).}
\label{fig:fft_line}
\end{figure}
To explore the frequency plane further we plot the intensity at $y$ frequency equals zero along the $x$ frequency axis (frequency plane of image shown in figure \ref{fig:fft_line} (a)). This describes the periodicity of the image in horizontal direction. As we see in figure \ref{fig:fft_line_cut} the intensity along this axis is constant.
\begin{figure}[H]
	\centering
		\includegraphics[width=0.8\textwidth]{pics/fft_simulation_cutoneline.pdf}
		\caption{The intensity of the Fourier transformed image in figure \ref{fig:fft_line} (a) at y frequency equals zero.}
		\label{fig:fft_line_cut}
\end{figure}
As a next step we want to find out what happens, if we insert a periodic signal into the image plane in the form of equally spaced lines. Figure \ref{fig:fft_lines} shows an image where there are several vertical lines with a spacing of 20 pixels. As in the case of the single line the image is constant in $y$ direction and so all vertical frequencies are assigned zero. In $x$ direction the lines create a periodic signal with frequency $\frac{1}{20} = 0.05 \frac{1}{\mathrm{px}}$. On the right side of figure \ref{fig:fft_lines} we see the fourier transformation of the image with the equally spaced lines and figure \ref{fig:fft_lines_cut} shows a horizontal cut through the center. We see that only the pixels at $0.05n \frac{1}{\mathrm{px}}$ $\forall n \in \{0, 1, 2, ...\}$ are non-zero.
\begin{figure}[H]
	\centering
		\includegraphics[width=1.0\textwidth]{pics/fft_simulationmorelines.pdf}
		\caption{The image with several equally spaced one pixel thick lines on the left is transformed to the frequency plane, see image on the right.}
		\label{fig:fft_lines}
\end{figure}
\begin{figure}[H]
	\centering
		\includegraphics[width=0.8\textwidth]{pics/fft_simulation_cutmorelines.pdf}
		\caption{The intensity of the Fourier transformed image in figure \ref{fig:fft_lines} at y frequency zero.}
		\label{fig:fft_lines_cut}
\end{figure}
In the appendix \ref{almostPeriod} we show, what happens if one line in the periodic image is missing and thus the signal is not completely periodic. We find that the treshold raises up to a higher value.\\
The spiders are not lines, but they have also an expansion into the horizontal, so we are also interested to see the Fourier transform of a single beam. We investigate the image of a beam (stair function) with a width of 10 pixels, placed at $x=10$. Figure \ref{fig:fft_beam} shows the corresponding image and its Fourier transform. As with the lines the only frequencies in the frequency plane with a non-zero value are along the $y=0 \frac{1}{\mathrm{px}}$ frequency axis. Figure \ref{fig:fft_beam_cut} shows this axis in more detail. In contrast to the frequency plane of the line we have a signal which is stronger for central frequencies and decreases slightly for larger frequencies. Additionally we have strong minima at $0.1 n$ $\forall n \in \mathbb{N}$, where the position of the minima is given by one over the width of the beam.
\begin{figure}[H]
	\centering
		\includegraphics[width=1.0\textwidth]{pics/fft_simulationonebeam.pdf}
		\caption{The image of one 10 pixel thick beam on the left is transformed to the frequency plane, see image on the right.}
		\label{fig:fft_beam}
\end{figure}
\begin{figure}[H]
	\centering
		\includegraphics[width=0.8\textwidth]{pics/fft_simulation_cutonebeam.pdf}
		\caption{The intensity of the Fourier transformed image in figure \ref{fig:fft_beam} at y frequency zero.}
		\label{fig:fft_beam_cut}
\end{figure}
As before with the lines we have a look at what happens if we have several of this beams, again equally spaced with a spacing of 20 pixels. We find that this image, lets call it $h(x)$, is a convolution of the image with the equally spaced lines $f(x)$ and the image with the single beam $g(x)$, namely
\begin{equation}
	h(x) = (f * g)(x) = \int_{\mathbb{R}^n} f(\tau)g(x-\tau) \mathrm{d}\tau .
\end{equation}
From the convolution theorem we find that for the Fourier transform it yields:
\begin{equation}
	\mathfrak{F}\{h(x)\} = \mathfrak{F}\{(f * g)(x)\} = (G \cdot F)(\mu),
\end{equation}
where $G(\mu)$ and $F(\mu)$ are the Fourier transforms of $g(x)$ and $f(x)$ \cite{ImageProcessing}. This means that the Fourier transform of the image with the several beams is given by the multiplication of the Fourier transform of the image with the equally spaced lines and the image with the single beam, which can be seen in figure \ref{fig:fft_beams_cut}. Around the center frequency we have some peaks separated by $0.05 \frac{1}{\mathrm{px}} = \frac{1}{20} \frac{1}{\mathrm{px}}$ which describes the separation between the beams of $20$ pixels. This peaks are surrounded by other peaks which are separated by $0.1 \frac{1}{\mathrm{px}} = \frac{1}{10} \frac{1}{\mathrm{px}}$ which describes the width of the beams of $10$ pixels.
\begin{figure}[H]
	\centering
		\includegraphics[width=1.0\textwidth]{pics/fft_simulationmorebeams.pdf}
		\caption{The image of several equally spaced 10 pixel thick beam on the left is transformed to the frequency plane, see image on the right.}
		\label{fig:fft_beams}
\end{figure}
\begin{figure}[H]
	\centering
		\includegraphics[width=0.8\textwidth]{pics/fft_simulation_cutmorebeams.pdf}
		\caption{The intensity of the Fourier transformed image in figure \ref{fig:fft_beam} at y frequency zero.}
		\label{fig:fft_beams_cut}
\end{figure}

\subsection{Gaussian Beams}
Before we have looked at beams with a stair function shape, but the spiders in our data do not have this stair function shape. In order to have a more realistic approximation we assume the spiders in our simulation to be Gaussian along the radial direction. We calculate the Gaussian profile from 
\begin{equation}
	f(x) = \exp \left(-\frac{(x-\mu)^2}{2 \sigma^2} \right)
\end{equation}
where $\mu$ is the mean (location) and $\sigma$ is the standard deviation (width). \\
Also we change to the image format of the warped image which is not quadratic but rectangular. Four our simulations we choose the radius range 254 to 454 pixels and compare it to the data from HD142527 in this range, which can be seen in figure \ref{fig:warped_254_454}. We have chosen the radius range such that the ghosts lie within it and one of the ghost is in the center of the radius range. So that latter on we can make sure, that we are able to suppress the signal from the spiders without loosing much of the ghosts which we use to simulate a really bright exoplanet. 
\begin{figure}[H]
	\centering
		\includegraphics[width=1.0\textwidth]{pics/warped_254_454.pdf}
		\caption{An image of HD142527 which is warped to the $r$-$\varphi$ plane and the radial intensity drop-off is subtracted.}
		\label{fig:warped_254_454}
\end{figure}
Figure \ref{fig:spider_gaussian} shows one of the spiders in a cutout from the image. We can see from the figure that a Gaussian is a good approximation for the shape of the spiders in radial direction.
\begin{figure}[H]
	\centering
		\includegraphics[width=1.0\textwidth]{pics/spyder_gaussian.pdf}
		\caption{A cutout of the image intensity of HD142527 at radius 255, where we see one of the spiders and a Gaussian profile at the same position as the spider is. We can see that the Gaussian is a good approximation for the shape of the spider.}
		\label{fig:spider_gaussian}
\end{figure}
We know that the Fourier transformation of a Gaussian profile is again a Gaussian profile and from the previous subsection we know that the Fourier transform does not depend on the location from the beam, but on the width of it. The spiders in our data all have different widths. In figure \ref{fig:Gauss_diffwidths} we plot four Gaussian profiles with different widths and their Fourier transforms. We see that indeed the Fourier transform of a Gaussian profile is also Gaussian (keep in mind that the plot is logarithmic) and that the width of the Gaussian profile is inverse proportional to the width of its Fourier transform as it already was the case for the stair function beams. Namely that the width of the Fourier transformed Gaussian is $w_{FFT} = \frac{1}{\sigma} \frac{1}{\mathrm{px}}$, where $w_{FFT}$ marks the x-position where the Fourier transformed is again zero. Also the intensity of the Fourier transform decreases with the width of the Gaussian profile, however this effect is rather small in the logarithmic plot. 
\begin{figure}[H]
	\centering
		\includegraphics[width=1.0\textwidth]{pics/Gauss_diffwidths.pdf}
		\caption{Gaussian profiles with different widths (top) and the respective Fourier transforms (bottom).}
		\label{fig:Gauss_diffwidths}
\end{figure} 
The VLT telescope produces four spiders which have a 180 degrees symmetry \cite{ESOmanual}. The spiders always have the same distance between each other, but the angular position and width can change from image to image. As a first approximation to this we have a look at four equal Gaussian beams which have the same spacing as in the data and fulfill the 180 degrees symmetry. Figure \ref{fig:Gauss_fourspyders} shows a horizontal cut through this setup and the respective frequency plane. We also plot the Fourier transform of a single Gaussian profile. Due to the linearity of the Fourier transformation we expect the Fourier transformation of the four equal Gaussian to be four times the Fourier transform of the single Gaussian profile. As we see from figure \ref{fig:Gauss_fourspyders}, where also a Fourier transformation of a single Gaussian profile (green line) is contained, this is also the case, but there is a strong oscillation and a beat as well. We do not understand why this is the case. Maybe it is a numerical problem, e.g. that the resolution is not high enough or maybe this happens because we only look at the real part of the Fourier transformation and the linearity only holds for the entire Fourier transform, meaning real and imaginary part.\\
When we average the Fourier transform over eight pixels in order get rid of the oscillations we get back a function in the frequency plane which is almost the same as the Fourier transform of a single Gaussian profile. Therefore it seems as if the oscillations include information about the distance between the various Gaussian profiles and it is probably not a numerical problem.
\begin{figure}[H]
	\centering
		\includegraphics[width=1.0\textwidth]{pics/Gaussian_fourspyders.pdf}
		\caption{The top shows a function with four Gaussian profiles which have a 180 degrees symmetry. The Fourier transform of this is shown in blue in the bottom image. In orange we have a averaged version of the Fourier transform (averaged over eight pixels). Additionally we plot in green the Fourier transform of a single Gaussian profile.}
		\label{fig:Gauss_fourspyders}
\end{figure} 

\subsection{Point-Like Sources}
In order to make sure that we are not going to suppress the signal of point-like sources as exoplanets we need to know how their signal is going to look like in the Fourier space.\\
We investigate the signal of a Gaussian point source and of a point spread function (PSF). A PSF describes how a point source looks like after it has gone through an imaging system \cite{PSFwiki}, which is in our case the VLT telescope with its adaptive optic. In order to simulate the PSF we used the python package AOtools \cite{AOtools}.\\
Figure \ref{fig:PSF_cut_image} shows a cut through the Gaussian and the PSF profile in the image plane. We want to compare the Fourier transformation of the PSF to the one of the Gaussian profile, of which we expect the Fourier transform to be again a Gaussian profile. The image with the PSF and its Fourier transform is shown in figure \ref{fig:PSF_fourier}. Figure \ref{fig:PSF_cut_fourier} shows the Fourier transform of a Gaussian profile and the PSF at radial frequency zero. As expected the Gaussian profile stays Gaussian in the frequency space. The Fourier transform of the PSF produces the same intensity for radial and angular frequency zero, but it decreases slower and has some local maxima. The two local maxima at each side of the global maximum produce a specific pattern which is completely different to what we get from the spiders and might be helpful to extract information about point sources from the frequency plane.  
\begin{figure}[H]
	\centering
		\includegraphics[width=1.0\textwidth]{pics/PSF_cut_image.pdf}
		\caption{A cut through a Gaussian profile and a PSF along the angular direction. Both profiles are rotationally symmetric.}
		\label{fig:PSF_cut_image}
\end{figure}
\begin{figure}[H]
	\centering
		\includegraphics[width=1.0\textwidth]{pics/PSF_fourier.pdf}
		\caption{An image with a PSF (top) is transformed into the frequency space (bottom).}
		\label{fig:PSF_fourier}
\end{figure}
\begin{figure}[H]
	\centering
		\includegraphics[width=1.0\textwidth]{pics/PSF_cut_fourier.pdf}
		\caption{The Fourier transform of a Gaussian profile and a PSF at radial frequency zero. }
		\label{fig:PSF_cut_fourier}
\end{figure}

\section{Suppress the Spiders using FFT}
\label{sec:suppression}
We now want to use the knowledge we have gained about the Fourier transformation of different features, especially of features which resemble spiders, to suppress the signal of the spiders. The idea is to manipulate the Fourier transform of the data and then transform the result back via the inverse Fourier transformation.\\
We take the data from HD142527, warp it to the $r$-$\varphi$ plane and correct for the radial intensity drop. For the transformation we choose the radius range as in figure \ref{fig:warped_254_454} namely: $R=254-454$, such that the object we want to look at is placed in the center. In order to make sure that the aperture flux of other objects than the spiders, mainly point sources like exoplanets, is conserved we will use the ghost at radius $323$ as a reference. The ghost is not perfectly in the center of the radius range, but this is not important for our purpose. The only thing which will happen, is that the aperture flux will change a bit when warping the image to the $r$-$\varphi$ plane.\\
Figure \ref{fig:rad0} shows the Fourier transform of the warped and flattened image at different radial frequencies. In contrast to the Fourier transform of the simulated spiders, the signal is not Gaussian anymore, since there are other features on top, like the ghost and the noise.
\begin{figure}[H]
	\centering
		\includegraphics[width=1.0\textwidth]{pics/rad0.pdf}
		\caption{Several horizontal cuts through the frequency plane of the image data shown in figure \ref{fig:HDsuppcentralfreq_R254_R454_-0.5to0.5}(a).}
		\label{fig:rad0}
\end{figure} 

\subsection{High Frequencies}
As we saw in section \ref{sec:pointlike} the angular range in which a point-source, Gaussian or PSF, will produce a significant frequency signal is within $[-50, 50] \frac{1}{\mathrm{rad}}$ and the radial range is within $[-0.2, 0.2] \frac{1}{\mathrm{px}}$. When we consider the noise, the range is even smaller as we saw in section \ref{sec:noise}.  
Therefore we first thought it would be a good idea to suppress everything which is outside of this area by a factor of $1000$. However it turned out that this is not the best idea, since the high frequencies are responsible for the edges in the image. By suppressing the high frequencies we would therefore blurry the image. This might not look like being a problem, since due to the noise the image is already blurred. As long as the point-source we look at is sufficiently large and bright this is indeed no problem, but as soon as the point-source gets fainter and smaller, it starts to disappear due to the blurring. Since we are more interested in objects of the second kind it is no option to suppress the high frequencies. Also we would not gain any positive effect through this suppression.\\
We also need to keep in mind here that we are plotting the absolute values of the frequency space and thus ignore big parts of the phase information as well as negative values. But we should not ignore this fact, when we do the suppression.\\
An other attempt tried was to subtract the mean of the noisy frequency background from the whole frequency space, but this was an even worse idea. Due to the subtraction the whole back transformed image gets a weaker intensity. All structures stay unchanged, but are weaker. Like this it would be even harder to find faint structures, especially if they come too close to the numerical limits. 

\subsection{Subtraction}
\label{sec:suppression_subtraction}
In order to be able to suppress the spiders, we need to understand them. As we learned in section \ref{sec:fourier} the spiders itself can be created by a convolution of the image containing the position and separation information, this means one pixel thick lines at the position of the spiders, and the image of a spider which contains the shape information, as the intensity and the width.\\
The spiders are added on top of the image. So the only option to suppress them completely is to subtract them, which is due to the linearity of the Fourier transformation also a subtraction in the frequency space. If we would do a division in the frequency space instead, which corresponds to unfolding the convolution. We would still remain with some parts of the information and maybe destroy image information, because the spiders are the result of first a convolution and then an addition.  This means we need to subtract the signal in the frequency plane generated by the spiders.\\
Since we do a subtraction we cannot only look at the absolute frequency values, we need to consider the complete real part. This means that we cannot ignore the position information of the spiders, which makes it a lot more complicated. Already now we can say that suppressing the spiders by subtracting certain frequencies in the Fourier space is a bad idea. Due to the fact that the point sources and the spiders are both located at the central frequencies we need to know exactly how the frequency signal produced by the spiders looks like. As we saw in \ref{sec:Fourier_spiders} this signal is complex and depends on the separations, the positions, the intensities and the widths of the spiders. We need to know all these parameters in order to be able to subtract the right frequencies. However if we know all these parameters precisely we can also subtract the spiders in the image plane and do not have to go to the frequency plane. Although we already know that it is not a good idea, we will still try it out. \\
\begin{figure}[H]
	\centering
		\includegraphics[width=1.0\textwidth]{pics/Spider_Gaussianfit_FFT.pdf}
		\caption{We fit Gaussian profiles to the mean intensity of the spiders in the central radial range (top). The function resulting from the fits is Fourier transformed (bottom).}
		\label{fig:Spider_Gaussianfit_FFT}
\end{figure}
To find the frequency spectrum which we have to subtract from the central radial frequency, which is the most important one (see section \ref{sec:fourier}), we sum up the central radial frequencies and fit Gaussian profiles to the spiders, see figure \ref{fig:Spider_Gaussianfit_FFT}. The Fourier transformation of the 1 dimensional fit has the same shape as the Fourier transformation of the spiders for radial frequency zero, the only difference is the intensity. We know that the central frequency is equal to the total intensity of the image, we therefore sum up the total intensity of the spiders only and use this information to adapt the intensity of the simulated Fourier transform. Then we subtract it from the frequency space of the data. The resulting frequency space and its back-transformed image are shown in figure \ref{fig:HDsubtraction}(b). To get back into the $r$-$\varphi$ plane we use the inverse Fourier transformation. Figure \ref{fig:HDsubtraction}(a) shows the warped and flattened image from $R=254-454$ and its Fourier transform. \\
When comparing the image before and after the subtraction, we see that around the central radial range the spiders have almost completely vanished. However we need to keep in mind that this was only possible by fitting Gaussian profiles to the spiders and if an object is really close to a spider it will be subtracted away, because the fit will count it to the spider. \\
From the steps we have made so far the aperture flux increases by $0.67$ \% compared to the initial aperture flux for ghost 2. We also insert a PSF close to the spider, where ghost 2 is, but in the central radius. If we place the PSF inside the spider it vanishes due to the subtraction, but if we place it next to the spider its aperture flux increases by $278$ \% for the PSF, here we use a PSF which is approximately $10^{-6}$ times less bright then the star. 
\begin{figure}[H]
	\centering
		\subfigure[]{\includegraphics[width=1.1\textwidth]{pics/HDflatten_R254_R454_-0.5to0.5.pdf}}
		\subfigure[]{\includegraphics[width=1.1\textwidth]{pics/HDsubtracted.pdf}}
\caption{An image from HD142527 which has been warped to the $r$-$\varphi$ plane and flattened and its Fourier transform (a). After a subtraction of a specific pattern at radial frequency zero, the spiders around the central radius are almost completely vanished(b).}
\label{fig:HDsubtraction}
\end{figure}

\subsection{Division}
\label{sec:suppression_division}
At the beginning of this work we accidentally did a division in the frequency space instead of a subtraction and we only realized this mistake at the very end of the thesis. As we already discussed in section \ref{sec:suppression_subtraction} division is not the right way to completely suppress the spider. Still we can achieve that the spiders become really narrow and less bright through a division in the frequency space. The only problem is that we also change the rest of the image data and not only regions around the spiders. Depending on how large this change is, the possibility of dividing could still be a good option. The division has the advantage that it is a lot more stable, meaning that we do not need to know the exact shapes and intensities of the spiders. Additionally we do not need to know the position information, since we do not change the sign by dividing by a positive number. \\

\subsubsection{Suppress Central Radial Frequencies}
\label{sec:sup_cen_radialfreq}
From our investigation of the Fourier transformation for spider-like structures, we found that their most important features are where the radial frequency is zero. Therefore, we first want to suppress the frequencies, which are caused by the spiders, at radial frequency zero.\\
As we saw in section \ref{sec:gaussian} the spiders produce a signal in the frequency plane, which is close to a Gaussian, this is especially the case when the radial frequency is equal to zero. We divide our central radial frequency by the Gaussian profile with width $\sigma = 8.7 \frac{1}{\mathrm{rad}}$, which we found from our simulations of the spiders, see figure \ref{fig:simspi_noise_angularfreq}. By doing this we ignore the fact that we have some oscillations on top of the true signal. But this is totally fine, since we only aim to transform the spiders into faint lines and the oscillations contain the separation information between the different spiders, which we will not need. Still we reassure that this approximation is appropriate by comparing the resulting back-transformed images after a division by the Gaussian profile and after a division by the Fourier transform at radial frequency zero which we get from our simulations. From the comparison we find that the Gaussian profile is a really good approximation, since by eye we cannot see any differences between the resulting back-transformed images and also the aperture fluxes of the ghosts are almost the same. We also compute the aperture flux of a PSF which we place on the image far away from the spiders and also this aperture flux is almost the same, also for different intensities of the PSF.\\
As a next step we want to find out in which angular range around the central frequency we need to divide by the Gaussian. This is a really important parameter because, if the angular width is so large, that the value of the Gaussian profile at this position is smaller than one, the division will lead to an increase of the intensity instead of a suppression. Figure \ref{fig:rad0_diffsubwidths} shows the aperture flux of the ghost after the division by a Gaussian at radial frequency zero for different angular frequency ranges around zero. The orange line marks the aperture flux of the ghost after the warping and the correction for the radial drop-off, we call it the initial aperture flux. Whereas the blue dots mark the aperture flux of the ghost after the division by the Gaussian for different angular widths.\\
Since it is an advantage, if the aperture flux increases through the procedure, we choose the width with which we get the largest aperture flux, which is the case for $9.1 \frac{1}{\mathrm{rad}}$. This means we divide by the Gaussian in the following angular range: $[-9.1, 9.1] \frac{1}{\mathrm{rad}}$.\\
An increase of the aperture flux is really good, because it means that the spiders are being suppressed without suppressing the ghost. Ghost 2, at which we are looking, is a good indicator for this, because in this image it is situated close to a spider.
\begin{figure}[H]
	\centering
		\includegraphics[width=1.0\textwidth]{pics/rad0_diffsubwidths.pdf}
		\caption{For radial frequency zero the Fourier transform of the image data is divided by a Gaussian which has its center at angular frequency zero. The division is not performed along the complete angular range, but only for a certain width around zero. We have plotted the aperture flux of ghost 2 for different angular division widths. The width which produces the largest flux is ideal.}
		\label{fig:rad0_diffsubwidths}
\end{figure}
Other parameters in this suppression are the intensity and the width of the Gaussian profile. We find that the resulting back-transformed image is not at all sensible on this variables. This is due to the fact that we divide by the Gaussian and do not subtract it. An other reason for the insensibility on the width of the Gaussian profile is that we only look at a narrow angular range, where the width of the Gaussian profile does not have a big influence. \\
Figure \ref{fig:HDsubtraction}(a) shows the warped and flattened image from $R=254-454$ and its Fourier transform. In figure \ref{fig:HDsuppcentralfreq_R254_R454_-0.5to0.5} the Fourier transform which is divided by the Gaussian profile at radial frequency zero in the above defined angular range and its back-transformed image are shown. To get back into the $r$-$\varphi$ plane we use the inverse Fourier transformation. When we compare the image before and after the Gaussian division we can already observe that the spiders are a lot less bright. Especially around the central radius the results are good. This region is the most important for us, since it is the only region, where the warping does not change the shape of the objects. Below the central radius the spiders almost stay the same or change only slightly and above the central radius we have a trend to the other extreme, meaning we have negative values. We also observe that the noise gets weaker and is smoother distributed.\\
From the steps we have made so far the aperture flux increases by $6.64$ \% compared to the initial aperture flux for ghost 2 and by $49.73$ \% for the PSF, here we use a PSF which is approximately $10^{-6}$ times less bright then the star. 
\begin{figure}[H]
	\centering
		\includegraphics[width=1.1\textwidth]{pics/HDsupprcentralfreq_R254_R454_-0.5to0.5.pdf}
		\caption{An image from HD142527 which has been warped to the $r$-$\varphi$ plane and flattened and its Fourier transform (a). After a division of the central frequencies at radial frequency zero by a Gaussian profile, the spider in the back-transformed image are less bright (b).}
\label{fig:HDsuppcentralfreq_R254_R454_-0.5to0.5}
\end{figure}

\subsubsection{Suppress Low Frequencies}
As we saw during the simulation of the spiders, not only the frequencies at radial frequency zero have non-zero values, but there is also an extension into the radial frequencies with non-zero values. In the following we will also suppress the frequencies which are caused by the spiders in this regime by using division. But due to the noise level, the interesting radial regime only goes until approximately $[-0.06, 0.06] \frac{1}{\mathrm{px}}$. As we already discussed in section \ref{sec:sup_cen_radialfreq} the regime is even smaller, since we need to avoid to divide through values which are smaller than one. The same holds for the angular range. The angular range is smaller for larger absolute radial frequency values, since the intensity of the complete signal is overall smaller. In conclusion we will only be able to suppress the lowest frequencies.\\
First of all we have to choose a Gaussian profile with an adequate width and intensity. As before when we suppressed the central radial frequency, we choose the Gaussian profiles such that it best fits to the signal of our spider simulation. For the signal at radial frequency $-0.005$/$0.005 \frac{1}{\mathrm{px}}$ (these are the ones closest to the radial frequency center and they are the same due to the symmetry of the Fourier transform) we choose a Gaussian profile with width $\sigma_s = 6.4 \frac{1}{\mathrm{rad}}$ and intensity $I_s = 0.84 I = 8.4 \cdot 10^3$ where $I$ is the intensity of the Gaussian profile used for suppressing the frequencies at radius zero. As before in section \ref{sec:sup_cen_radialfreq} the width and the intensity of the Gaussian profile are not so important for the suppression and it does not matter if they differ a bit. But the angular frequency width $w_s$ in which we do the division (as in section \ref{sec:sup_cen_radialfreq}) as well as the radial frequency width $h$ is essential.\\
To find these two parameters we use the same procedure we used in section \ref{sec:sup_cen_radialfreq} to find the angular frequency width $w$ for radial frequency zero. Namely we try out different parameters and take the one with which we get the largest aperture flux for ghost 2. As before we will also use a second object in form of a PSF which is placed not in the surrounding of one of the spiders, in order to check does not change the background in unwanted way. The PSF is approximately $10^{-7}$ times less bright then the star. We find that $h = 0.01 \frac{1}{\mathrm{px}}$ and $w_s = 7.2 R \frac{1}{\mathrm{rad}}$, where $R$ is the ratio between the mean central radial frequency (considering only low angular frequencies) and the mean radial frequency in which the division is taking place.\\
The parameters for the other radial frequencies with an absolute value which is larger than $0.005 \frac{1}{\mathrm{px}}$, but still smaller than $h$, are derived from the one at absolute radial frequency $0.005 \frac{1}{\mathrm{px}}$. In figure \ref{fig:simspi_angularfreq} we observed that the non-central low radial frequencies all have the same shape, but different intensities. Therefore we choose the same Gaussian profile for all of them and multiply it by factor $R$.\\
Figure \ref{fig:HDsupplowfreq_R254_R454_-0.5to0.5.pdf} shows the image from HD142527 after the suppression of the low frequencies and the suppression of the central radial frequency. We observe that the overall brightness of each spider from figure \ref{fig:HDsuppcentralfreq_R254_R454_-0.5to0.5} to figure \ref{fig:HDsupplowfreq_R254_R454_-0.5to0.5.pdf} has not changed significantly, but the gradient from lower to larger radii which we had in figure \ref{fig:HDsuppcentralfreq_R254_R454_-0.5to0.5} (b) has almost completely vanished. Also the whole image is a lot smoother. This can also be observed in the aperture fluxes. As ghost 2 is placed near a spider and is almost at the center of the radial range, its aperture flux is almost the same as after the suppression of the central radial frequency only and so the aperture flux increased by $6.74$ \% compared to its initial aperture flux, whereas the aperture flux of the PSF increased by $53.12$ \%. 
\begin{figure}[H]
	\centering
		\includegraphics[width=1.1\textwidth]{pics/HDsupplowfreq_R254_R454_-0.5to0.5.pdf}
		\caption{The image from HD142527 after the central radial frequencies and the low frequencies are both suppressed in the frequency plane and transformed back to the image plane via inverse Fourier transformation.}
		\label{fig:HDsupplowfreq_R254_R454_-0.5to0.5.pdf}
\end{figure}

\section{Conclusion}
To suppress the spiders in astronomical data we used Fast Fourier transformation. In order to do so it is a good method to warp the image data to the $r$-$\varphi$ plane. In this plane the radial intensity structures are then all parallel to the new $y$-axis as the spider and now that the radial intensity drop-off due to the star can be filtered out easily.\\
In our work we used data from HD142527 and we look at the radial regime from $R=254-454$. In this regime the two ghosts are situated, we calculated there aperture flux to check that our suppression did not accidentally filter out round objects. Additionally we also inserted a simulated PSF to investigate the effect on fainter objects. \\
We decide to suppress the spiders by dividing the low frequencies by a Gaussian profile. The advantage of the division is that we can use the Gaussian profile as an approximation of the real signal, which is a lot more complicated. Additionally we can use the same intensity and width of the Gaussian profile for all images, since the suppression by division is not sensible on these parameters. However we need to limit the angular range, so that we do not divide by small numbers. An other problem of the division is that it is actually not the right approach, if we consider the theory. Dividing means that we treat the spiders as if they were a convolution on the image, but in reality they are added on top of the image data. This means that we should subtract the frequencies caused by the spiders. The problem about the subtraction is that we cannot do all these approximation, like approximating it by a Gaussian. Also the parameters are a lot more sensible and need to be chosen precisely. \\
Although the suppression by dividing the low frequencies by a Gaussian profile is theoretically not the right approach, we still receive a result, where the spiders are suppressed, at least a little and we do not loose much information of the objects in which we are interested. However the suppression only works well around the center of the radius range chosen. Also the resulting suppression is quite small, so that it is questionable if it makes sense to do it. If one decides to do suppression by division, then one should only suppress the central radial frequency, since the suppression by other frequencies does not lead to success.\\ 
It would make sense to try to suppress the spiders by subtracting certain frequencies in a future work, also if this means that one needs to determine the parameters really precisely and have individual parameters for each image data. 
 


\section{Acknowledgments}
I (Jennifer Studer) want to thank Hans Martin Schmid for is help and support I got from him during the entire project and the interesting conversation we had. For the supervision I want to thank Hans Martin Schmid and Christian Tschudi. I also want to thank Kaya Ercihan and Tanja Finger for proofreading parts of my thesis. 

\newpage 

\appendix
\section{FFT of an almost Periodic Signal}
\label{almostPeriod} 
In section \ref{FFT_Lines_Beams} we have seen the Fourier transform of an image with equally spaced lines. This is a periodic pattern. But what happens if the pattern is not completely periodic anymore? By taking out one of the lines we explore the effect on the Fourier transform. In figure \ref{fig:fft_lines_almostper} we see the image, where the line which should be at $x=90$ is missing. From figure \ref{fig:fft_lines_cut_almostper} we see that the frequency spectrum still has the same shape, but the intensity range changes. The background which before was at zero (in order to plot logarithmic we added a small value) is now lifted to an intensity of $100$. 
\begin{figure}[H]
	\centering
		\includegraphics[width=1.0\textwidth]{pics/fft_simulationmorelines_almostper.pdf}
		\caption{An image with several one pixel thick lines which are almost periodically distributed on the left is transformed to the frequency plane, see image on the right.}
		\label{fig:fft_lines_almostper}
\end{figure}
\begin{figure}[H]
	\centering
		\includegraphics[width=0.8\textwidth]{pics/fft_simulation_cutmorelines_almostper.pdf}
		\caption{The intensity of the Fourier transformed image in figure \ref{fig:fft_lines_almostper} at y frequency zero.}
		\label{fig:fft_lines_cut_almostper}
\end{figure}

\newpage
\bibliographystyle{plain}
\bibliography{bib_exo}

\end{document}
